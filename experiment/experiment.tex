%!TEX root = ../thesis.tex

\section{実験1}
\subsection{実験目的}
シミュレータ上で実験を行い, 提案手法の有効性を検証する.

\subsection{実験装置}
実験は, \figref{Fig:gazebo}に示すGazebo\cite{gazebo}のWillow Garage\cite{willow}で\figref{Fig:willow-garage}に示すコースで一周行う.
 また, ロボットモデルには\figref{Fig:turtlebot3}に示すようなカメラを3つ搭載したTurtlebot3\cite{turtlebot3}を用いた. 

\begin{figure}[h]
  \centering
  \includegraphics[keepaspectratio, scale=0.15]{images/gazebo.png}
  \caption{Experimental environment in simulator}
  \label{Fig:gazebo}
  \end{figure}

\newpage
\vspace{20mm}
\begin{figure}[h]
  \centering
  \includegraphics[keepaspectratio, scale=0.5]{images/willow-path.png}
  \caption{Course to collect data}
  \label{Fig:willow-garage}
  \end{figure}

\begin{figure}[h]
  \centering
  \includegraphics[keepaspectratio, scale=0.55]{images/turtlebot3.png}
  \caption{Turtlebot3 waffle with 3 cameras}
  \label{Fig:turtlebot3}
  \end{figure}

\newpage
\subsection{実験方法}
\begin{description}
  \item[1.データ収集フェーズ]\mbox{}\\データの収集方法について述べる. 
  \figref{Fig:willow-garage}に示す目標経路に沿うよう自律移動させ, 0.2m走行するごとに, 
  カメラ画像とナビゲーションの出力である角速度を収集する. 
\end{description}

\begin{description}
  \item[2.訓練フェーズ]\mbox{}\\データ収集フェーズで収集したデータ1935個を用いて, 
  バッチ数16, エポック数150のミニバッチ学習で学習した. 
\end{description}

\begin{description}
  \item[3.テストフェーズ]\mbox{}\\ \figref{Fig:willow-garage}に示すコースで100個の学習済みモデルを使用して走行させる. 
  ロボットの並進速度0.2m/sとし, 経路を1周できた場合を成功, 壁に激突したり, 経路から10m離れたりした場合を失敗とした.
\end{description}

\subsection{実験結果}
実験結果を表\ref{tb:exp1}に示す. また, 失敗箇所は\figref{Fig:result1}のようになった. 
\par \figref{Fig:result1}の×の箇所で曲がり切ることができずにコースアウトしてしまった. 
訓練時のlossを\figref{Fig:exp1_loss}に示す. 
図では. 学習が収束している様子が確認できる. ここで, 角を曲がりきれなかった要因の一つとして, 
目標経路周辺のデータが足りないためだと考えられる. 
例えば先行研究では目標経路に沿った向きを基準として平行に±0.20m離した上で±5度傾けた際のデータも収集している
これを踏まえて, 同じように平行に±0.20m離した上で±5度傾けた際のデータを走行しながら収集し, 
データ数を増やすことで成功回数が増えるか検証する. 

\begin{table}[h]
  \centering
  \caption{Number of successes in the batch learning}
  \begin{tabular}{|c|c|} \hline0
    Experiments & Number of successes \\ \hline
    Exp.1(epoch150) & 42/100 \\ \hline
  \end{tabular}
  \label{tb:exp1}
\end{table}

\begin{figure}[h]
  \centering
  \includegraphics[keepaspectratio, scale=0.5]{images/result1.png}
  \caption{Failure point of the experiment}
  \label{Fig:result1}
  \end{figure}

\vspace{20mm}
\begin{figure}[h]
  \centering
  \includegraphics[keepaspectratio, scale=0.5]{images/exp1.2_4000.png}
  \caption{Loss value in the experiment1}
  \label{Fig:exp1_loss}
  \end{figure}

\newpage
\section{実験2}
実験目的, 実験装置, テストフェーズは実験1と同様である.
\subsection{実験方法}
\begin{description}
  \item[1.データ収集フェーズ]\mbox{}\\実験1を踏まえて, 経路周辺のデータを多く取得する手法を試みる. 
  \figref{Fig:collect-data}にデータの収集方法を示す. 赤色の線である目標経路から平行に±0.02mの位置に搭載したカメラに加え
  ヨー軸方向の画像も取得するため, ±5度傾けたカメラも搭載する
  角速度を\figref{Fig:collect-data2}のように収集する. これを\figref{Fig:willow-garage}に示すコースで一周行う. 
  なお, 走行しながら集める性質上, 角速度が中央のカメラとペアになるものしか取得できないため
  オフセットを各カメラの分用意し, 中央の角速度に加えることで再現する
  オフセットは先行研究で収集された1周分の角速度の平均とする
\end{description}

\begin{figure}[h]
  \centering
  \includegraphics[keepaspectratio, scale=0.18]{images/collect-data.png}
  \caption{Method of collecting data around the target route}
  \label{Fig:collect-data}
  \end{figure}

\begin{description}
  \item[2.訓練フェーズ]\mbox{}\\データ収集フェーズで収集したデータ5805個を用いて, 
  バッチ数16, エポック数200のミニバッチ学習で学習した. 
\end{description}

\newpage
\subsection{実験結果}
実験結果を表\ref{tb:exp2}に, 学習のlossを\figref{Fig:exp2_loss}に示す.
経路周辺の画像データも入力データに含むことでデータ数を増やし, ミニバッチ学習を用いて訓練することで経路追従できることを確認した. 

\subsection{検証}
実験2では 150/150 の試行がすべて成功し, 安定したモデルを生成できた.
また, 実験中にロボットを意図的に経路から逸脱させたところ, いずれの試行でも経路への復帰動作を確認した.
しかし, この復帰行動が経路上のすべての地点で同様に発生するのかについては不明である.そこで, この点をさらに検証することとした.
検証方法には先行研究で清岡らが使用した方法\figref{Fig:old-method_analysis}を参考に, 
一定間隔に置いた場所からモデルの出力を記録した
結果を\figref{Fig:this-method_analysis}に示す
どの地点でも経路から外れた際にはもとに戻ろうとする挙動が確認できた

\vspace{10mm}
\begin{table}[h]
  \centering
  \caption{Number of successes in the experiment}
  \begin{tabular}{|c|c|} \hline
    Experiments & Number of successes \\ \hline
    Exp.2(epoch200) & 150/150 \\ \hline
  \end{tabular}
  \label{tb:exp2}
\end{table}

\vspace{20mm}
\begin{figure}[h]
  \centering
  \includegraphics[keepaspectratio, scale=0.5]{images/exp1.2_4000.png}
  \caption{Loss value in the experiment2}
  \label{Fig:exp2_loss}
  \end{figure}

\newpage
\vspace{20mm}
\begin{figure}[h]
  \centering
  \includegraphics[keepaspectratio, scale=0.5]{images/exp1.2_4000.png}
  \caption{verification of the output of the learning machine}
  \label{Fig:old-method_analysis}
  \end{figure}

\vspace{20mm}
\begin{figure}[h]
  \centering
  \includegraphics[keepaspectratio, scale=0.5]{images/exp1.2_4000.png}
  \caption{verification of the output of the learning machine}
  \label{Fig:this-method_analysis}
  \end{figure}

\newpage
\section{実験3}
実験2では\figref{Fig:9camera_Layout}のように平行に±0.20m離した上, 更に±5度の画像, 計9台のカメラを使用しているが, 
実環境を考えると現実的ではないため, ±5度の画像は射影変換によって疑似的に再現する. 
実験目的, 実験装置, データ収集フェーズ, テストフェーズは実験2と同様である. 

\subsection{実験方法}
\begin{description}
  \item[2.訓練フェーズ]\mbox{}\\データ収集フェーズで収集したデータ1935個を用いて, 
  バッチ数16, エポック数200のミニバッチ学習で学習した. 
  なお, 射影変換に使用する焦点距離fは$f_x = \frac{W_{\text{px}}}{2 \tan\left(\frac{\mathrm{FOV_h}}{2}\right)}$を使う.
  今回水平視野角は2.09rad, 画像サイズは64*48である
\end{description}

\subsection{実験結果}
実験結果を表\ref{tb:exp3}に示す.

\vspace{10mm}
\begin{table}[h]
  \centering
  \caption{Number of successes in the experiment}
  \begin{tabular}{|c|c|} \hline
    Experiments & Number of successes \\ \hline
    Exp.3(4000step) & 150/150 \\ \hline
  \end{tabular}
  \label{tb:exp3}
\end{table}

\vspace{20mm}
\begin{figure}[h]
  \centering
  \includegraphics[keepaspectratio, scale=0.5]{images/exp1.2_4000.png}
  \caption{verification of the output of the learning machine}
  \label{Fig:9camera_Layout}
  \end{figure}