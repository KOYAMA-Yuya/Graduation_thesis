%!TEX root = ../thesis.tex

\newpage
\section{目的}

本論文では,岡田らの手法において問題となっていた学習に要する時間を短縮することを目的とし,
高橋らによって提案されたオフラインによる経路追従行動の模倣学習手法を基に,
ロボット走行中に複数カメラ画像と行動データを同時に収集可能なシステムを構築する.
そして,構築したシステムを用いることで,学習時間の短縮と経路追従性能の両立が可能であるかを検証する.

オフライン模倣学習は他の研究においても行われているが,
本研究で提案する手法は,データセットを自動的に収集できる点に特徴がある.
さらに,実ロボットへの適用を念頭に置き,
経路周辺の視覚情報が経路追従行動に与える影響について,
シミュレータを用いた実験により明らかにすることも目的とする.

\section{論文構成}

本論文の構成は以下に述べる通りである. 第1章では, 研究を行う背景や目的を述べた. 第2章では, 研究に関連する要素技術, 
第3章では, 従来手法について説明する. 第4章では, 提案手法について説明し, 第5章では, 実験について説明する. 
そして, 第6章では, 本研究の結論を述べる. 
