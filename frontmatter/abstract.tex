%!TEX root = ../thesis.tex
\chapter*{概要}
\thispagestyle{empty}
%
\vspace{-7mm}
% 
\begin{center}
  \scalebox{1.5}{視覚と行動のend-to-end学習により経路追従行動を}\\
  \scalebox{1.5}{オンラインで模倣する手法の提案}\\
  \scalebox{1.3}{(学習用データを走行中に自動収集するシステムの構築と検証)}
\end{center}
%

本論文では, 髙橋らによって提案された, 事前に収集した画像と行動を用いて経路追従を学習するオフライン模倣学習手法に着目し, 
実環境への適用に向けたデータ収集システムの構築および有効性の検証を行う. 

髙橋らの手法は,オフライン学習による経路追従の可能性を示したものの, 
学習用データの収集はロボットを手動で配置し直す作業に依存しており, 走行中に自動でデータを蓄積するシステムは構築されていなかった.  
そのため, 多数の姿勢・位置条件を網羅するには多大な時間と労力を要し, 実環境への適用における大きな障壁となっていた. 
そこで本研究では, ロボットの走行中に複数カメラ画像と行動データを同期して自動収集可能なシステムを新たに構築した.  
これにより, 従来手法で課題であったロボットの置き直し作業を排除し, 効率的なデータ収集を可能とした. 
また, 実機実装におけるコストと設置負荷を考慮し, 従来研究の9台の物理カメラ構成を見直し, 
3台のカメラ画像から射影変換を用いて9台相当の視野情報を再構成する手法を導入した. 

シミュレータを用いた実験の結果, 本システムを用いることで, わずか1周の走行データから既存手法と同等の経路追従性能を実現し, 
学習にかかる作業時間を大幅に短縮できることを確認した.  さらに, 実環境における走行実験を通じて, 
提案システムの有効性を検証するとともに, 一部区間において経路追従が困難となる事例を確認した. 
これにより, 本手法の実環境適用に向けた課題が明らかとなった.


\vspace{10mm}
キーワード: end-to-end 学習, ナビゲーション, オフライン

%
\newpage
%
\chapter*{abstract}
\thispagestyle{empty}
%
\begin{center}
  \scalebox{1.3}{A proposal for an online imitation method of path-tracking}
  \scalebox{1.3}{behavior by end-to-end learning of vision and action}
  \makebox[\textwidth][c]{\scalebox{1.}{(Development and validation of an automated training data collection system during driving)}}
\end{center}

%
This paper focuses on the offline imitation learning method for path-tracking proposed by Takahashi et al. and proposes an automated data collection system designed for real-world applications. Although the previous method demonstrated the potential of offline learning, it relied heavily on manual robot repositioning to gather training data. This dependency made it labor-intensive to cover a comprehensive range of poses and positions, posing a significant barrier to practical implementation.

To overcome this, we developed a new system capable of automatically synchronizing and collecting multi-camera images and action data while the robot is in motion. To reduce the hardware cost and installation burden for real-world deployment, we revised the original nine-camera configuration. Instead, we introduced a method that uses projective transformation to reconstruct the equivalent field-of-view of nine cameras from only three physical cameras.

Simulation results confirmed that the proposed system achieves path-tracking performance comparable to existing methods using only a single lap of driving data, significantly reducing the effort required for training. Furthermore, real-world experiments validated the system’s effectiveness and clarified remaining challenges for its practical application.
\vspace{1mm}

keywords: End-to-End Learning, Navigation, Offline 