%!TEX root = ../thesis.tex

本章では, 従来研究を基にしたオフラインでデータを収集し訓練する手法を提案する.

\section{手法}
本研究で提案する手法を述べる. 従来手法に対して, 提案手法は自律走行しながら複数カメラにて一度にデータを収集し, 
オフラインで訓練することが異なる.  \par \figref{Fig:collect-data2}にデータの収集方法を示す. 
赤色の線である目標経路に沿うよう, ロボットを1周だけ自律移動させる
そして, 0.2mごとに64×48のカメラ画像(RGB画像)とルールベース制御器によるナビゲーションの出力である角速度を
\figref{Fig:collect-data2}のように収集する. ロボットの進行方向に対する並進速度は0.2m/sであるが, 
データセットにはナビゲーションの出力である角速度のみがロボットに与えられる. \par 
このように, ロボットを自律移動させることで, 自動で大量のデータを収集することができる. 
その後, 収集したデータを用いてオフラインでミニバッチ学習を行う. 

% \vspace{10mm}
\newpage
  \begin{figure}[h]
  \centering
  \includegraphics[keepaspectratio, scale=0.2]{images/collect-data.png}
  \caption{Method of collecting data around the target route}
  \label{Fig:collect-data2}
  \end{figure}