%!TEX root = ../thesis.tex

本論文では,  岡田らの手法において課題となっていた学習に要する時間の短縮を目的として,  
高橋らにより提案されたオフラインによる経路追従行動の模倣学習手法を基に,  
ロボット走行中に複数カメラ画像と行動データを同時に自動収集可能なシステムを構築した. 
本研究を通じて得られた知見を以下にまとめる. 

第一に,  ロボット走行中に複数台のカメラ画像と制御入力を同期して収集するシステムを構築したことで,  
従来手法で不可欠であったロボットの手動配置作業を排除した. 
これにより,  わずか1周の走行データから経路追従に必要なデータセットを網羅的に生成することが可能となり,  
学習準備に要する時間を大幅に短縮できることを示した. 

第二に,  実用性を考慮した視覚情報の構成について検証した. 9台の物理カメラを用いた構成に加え,  
3台の物理カメラ画像に射影変換を適用して視野情報を拡張する手法を導入した. シミュレータを用いた実験の結果,  
射影変換を用いた構成においても,  従来の多台数カメラ構成と同等の経路追従性能を維持できることを確認した. 
これにより,  ハードウェア構成の簡略化と効率的なデータ収集の両立が可能となった. 

以上の結果から,  本研究で構築したシステムは,  オフライン模倣学習におけるデータ収集の自動化を可能とし,  
学習効率と経路追従性能の両立に有効であることが示唆された. 
また,  シミュレーション環境を用いた評価を通して,  経路周辺の視覚情報が経路追従行動に与える影響を明らかにし,  
実ロボットへの適用に向けた有用な知見を得た. 